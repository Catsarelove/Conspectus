\documentclass {article}
\usepackage[usenames]{color}
\usepackage{tikz}
\usepackage{colortbl}
\usepackage{ulem}
\usepackage[pdfborder=0 0 0,colorlinks,linkcolor=black]{hyperref}
\usepackage{verbatim}
\usepackage[utf8]{inputenc}
\usepackage[T1, T2A]{fontenc}
\usepackage[english,russian]{babel}
\usepackage{indentfirst}
\usepackage{listings}
\usepackage{titling}
\usepackage{geometry}
\geometry{pdftex, a4paper}
\geometry{left=2.5cm, right=2.5cm, top = 2cm}
\usepackage{caption}
\DeclareCaptionFont{white}{\color{white}}
\definecolor{mycol1}{rgb}{0.58,0.78,0.2}
\DeclareCaptionFormat{listing}{\colorbox{mycol1}{\parbox{\textwidth}{#1#2#3}}}
\captionsetup[lstlisting]{format=listing}
\predate{}
\postdate{}
\date{}
\posttitle{\par\end{center}}
\title{{\bfseries Полиморфизм. Перегрузка}}
\begin{document}
\definecolor{mycol}{rgb}{0.98,0.92,0.9}
\maketitle
\section{Введение}
{\bfseries Полиморфизм} --- это способность функции обрабатывать данные разных типов
\begin{center}.
\begin{tikzpicture}
\path (-8, 1) node(x) {Полиморфизм}
(0, 3) node(y) {\textit{ad hoc} полиморфизм}
(0, 1) node(y1) {один и тот же код для всех допустимых типов аргументов}
(0,-1) node(z) {параметрический полиморфизм}
(0,-3) node(z1) {потенциально разный код для каждого типа или подтипа аргумента};
\draw [->, mycol1, thick] (x) -- (y);
\draw [->, mycol1, semithick] (y) -- (y1);
\draw [->, mycol1, semithick] (z) -- (z1);
\draw [->, mycol1, thick] (x) -- (z);
\end{tikzpicture}
\end{center}
\paragraph{Общие идеи}
\begin {enumerate}
\item Использование общего кода для разных типов
\item Унифицированный вид кода (одни и те же операторы для разных случаев)
\item Основные языковые инструменты:
	\begin {itemize}
		\item приведение типов
		\item указатели на функции
		\item перегрузка функций
		\item шаблоны
	\end {itemize}
\end {enumerate}
\paragraph{Приведение типов\\}
Бывает {\bfseries явное} и {\bfseries неявное}. Неявное преобразование выполняет компилятор, а явное --- сам программист
\lstset{%
language=C,            
basicstyle=\small\ttfamily, 
numbers=left,       
numberstyle=\tiny,       
stepnumber=1,           
numbersep=5pt,            
backgroundcolor=\color{mycol},
showspaces=false,  
showstringspaces=false,  
showtabs=false,  
keywordstyle=\bfseries\color{green!40!black},
identifierstyle=\color{blue},
stringstyle=\color{orange},   
frame=single,    
tabsize=2,         
captionpos=t,       
breaklines=true,         
breakatwhitespace=false,
escapeinside={\%*}{*)}   
}
\begin{lstlisting}[caption=Неявное приведение типов]
double x = 5.0;
int y = x; 
\end{lstlisting}
\begin{lstlisting}[caption=Явное приведение типов]
double x = 5.0;
int y = (int)x;
\end{lstlisting}
\begin{lstlisting}[caption=Явное приведение типов]
int ret=15;
static_cast<float>(ret)/2;
\end{lstlisting}
\paragraph{Указатели на функции}
\begin {itemize}
	\item Функции тоже могут быть аргументами других функций. 
	\item Можно применить функцию ко всему списку/дереву/массиву 
\end {itemize}
\begin{lstlisting}[caption=Реализация указателя на функцию]
void f(int x) { }
void (*pf)(int) = &f;
pf(8);
\end{lstlisting}
\section{Перегрузка}
\subsection{Перегрузка функций}
{\bfseries Перегрузка функций (overloading)} --- поддержка нескольких функций с одинаковым именем, но разными аргументами.\\
При компиляции к именам функций добавляются аргументы и длина имени функици\\
\texttt{foo(int a, float b)} будет представлено как \texttt{\_Z3fooif }\\
Компилятор сам выбирает, что подставить по аргументам.
\begin{lstlisting}[caption=Пример перегрузки функции]
int sum(int a, int b)         {return a+b;}
float sum(float a,float b)    {return a+b;}
float sum(float a,int b)        {return sum(a,(float)b);}
\end{lstlisting}
Причем, например, функцию ‘‘ {\bfseries int} sum({ \bfseries \textcolor{red}{ float}} a, {\bfseries \textcolor{red}{ int}} b) '' добавить уже нельзя.\\
\def\tabrowsep{\noalign{\vskip 1.2pt}}
\\Что можно использовать для перегрузки функции?
\begin{center}
\begin{tabular}{c c}
\rowcolor[rgb]{0.58,0.78,0.2} Элемент объявления функции & Использование для перегрузки\\\tabrowsep
\rowcolor[rgb]{0.7,0.7,0.7} Тип возвращаемого функцией значения & Нет \\\tabrowsep
\rowcolor[rgb]{0.7,0.7,0.7} Число аргументов & Да \\\tabrowsep
\rowcolor[rgb]{0.7,0.7,0.7} Тип аргументов & Да \\\tabrowsep
\rowcolor[rgb]{0.7,0.7,0.7} Наличие или отсутствие многоточия & Да \\\tabrowsep
\rowcolor[rgb]{0.7,0.7,0.7} Использование имен typedef & Нет \\\tabrowsep
\rowcolor[rgb]{0.7,0.7,0.7} Незаданные границы массива & Нет \\\tabrowsep
\rowcolor[rgb]{0.7,0.7,0.7}{\bfseries const} или {\bfseries volatile}$^1$ & Да \\
\end{tabular}
\end{center}
\footnotesize{$^1$ Используются для перегрузки, только если они применяются в классе для указателя {\bfseries this}, а не для типа возвращаемого значения функции. Перегрузка применяется, только если ключевое слово {\bfseries const} или {\bfseries volatile} следует за списком аргументов функции в объявлении}\\
\normalsize
\subsection{Перегрузка методов}
{\bfseries Перегрузка методов} --- поддержка нескольких методов с одинаковым именем, но разными аргументами.
\begin {itemize}
	\item Методы могут быть устроены по-разному в зависимости от аргументов
	\item Пример --- конструкторы
\end {itemize}
\begin{figure}[ht]
\begin{lstlisting}[caption=Перегрузка метода]
Class Matrix{
    double** a;
public:
...
    Matrix Multiply(double b);
    Matrix Multiply(Vector b);   
    Matrix Multiply(Matrix b);    
};
\end{lstlisting}
\end{figure}
\subsection{Перегрузка функций в классе}
\begin {itemize}
	\item Можно перегрузить функции для своего класса
	\item Можно перегрузить функцию, имеющую доступ к private-полям класса - дружественную
\end {itemize}
\begin{lstlisting}[caption=Перегрузка функции swap]
void swap(MyClass& A, MyClass& B);
\end{lstlisting}
\subsection{Перегрузка операторов}
Операторы транслируются компилятором в функции(методы) и поэтому тоже могут быть перегружены. \\
\textcolor{red}{Исключения:} \texttt{<.*>,<.>,<?:>} и \texttt{<::>}.
\begin {itemize}
	\item При перегрузке сохраняется арность оператора (например, бинарный оператор не может стать унарным)
	\item Общий вид типов для перегрузки есть в стандарте
	\begin{itemize}
		\item Поскольку из объекта виден только интерфейс класса, обычно довольно строго смотрят на константность аргументов
	\end{itemize}
	\item Переопределение операторов для стандартных типов тоже невозможно
	\begin{itemize}
		\item можно создать свой тип-синоним: \texttt{\textcolor{blue}{typedef int} MyType}, и с ним делать, что нужно.
	\end{itemize}
	\item Перегрузка оператора в классе может быть:
	\begin{itemize}
		\item внешней (фунцкия) для бинарных операторов
		\item внутренней (метод) для унарных операторов
	\end{itemize}
\end {itemize}
При перегрузке сравнений обычно определяется минимальный набор, а остальные операторы через определенные, например \texttt{==} и \texttt{>} --- основа.
\begin{lstlisting}[caption=Перегрузка сравнений]
bool operator==(MyType a, MyType b){some code}
bool operator>(MyType a, MyType b){some code}
bool operator>=(MyType a, MyType b){return (a == b)||(a > b);}
bool operator!=(MyType a, MyType b){return !(a == b);}
bool operator<(MyType a, MyType b){return (!(a == b) || !(a > b));}
\end{lstlisting}
\begin{lstlisting}[caption=Перегрузка инкремента и декремента]
MyType& operator++();            // prefix - ++a
MyType& operator++(int a);           // postfix   - a++
\end{lstlisting}
Аргумент в постфиксной форме может быть использован, но это нежелательно. Нужен только для различий имен функций при трансляции.\\
Оператор индексирования \texttt{[]} всегда определяется как член класса и, так как подразумевается поведение индексируемого объекта как массива, то ему следует возвращать ссылку.
\begin{center}
\begin{tabular}{c c}
\rowcolor[rgb]{0.58,0.78,0.2} Оператор & Рекомендуемая форма\\\tabrowsep
\rowcolor[rgb]{0.7,0.7,0.7} Все унарные операторы & Член класса \\\tabrowsep
\rowcolor[rgb]{0.7,0.7,0.7} \texttt{= () [] -> ->*} & Обязательно член класса \\\tabrowsep
\rowcolor[rgb]{0.7,0.7,0.7} \texttt{+= -= /= *= \^}\texttt{= \&= |= \%= >>= <<=} & Член класса \\\tabrowsep
\rowcolor[rgb]{0.7,0.7,0.7} Остальные бинарные операторы & Не член класса\\\tabrowsep
\end{tabular}
\end{center}
Стоит использовать следующие способы способы передачи аргументов в функции и возвращения значений операторов:
\begin{itemize}
\item Если аргумент не изменяется оператором, в случае, например унарного плюса, его нужно передавать как ссылку на константу. Вообще, это справедливо для почти всех арифметических операторов (сложение, вычитание, умножение...)
\item Тип возвращаемого значения зависит от сути оператора. Если оператор должен возвращать новое значение, то необходимо создавать новый объект (как в случае бинарного плюса). Если вы хотите запретить изменение объекта как l-value, то нужно возвращать его константным.
\item Для операторов присваивания необходимо возвращать ссылку на измененный элемент. Также, если вы хотите использовать оператор присваивания в конструкциях вида \texttt{(x=y).f()}, где функция \texttt{f()} вызывается для для переменной \texttt{x}, после присваивания ей \texttt{y}, то не возвращайте ссылку на константу, возвращайте просто ссылку.
\item Логические операторы должны возвращать в худшем случае \texttt{int}, а в лучшем \texttt{bool}.
\end{itemize}
\begin{lstlisting}[caption=Перегрузка специальных символов]
class vector{
...
    int operator [](size_t i) const {return array[i];}  
    int& operator [](size_t i) {return array[i];}     
};
\end{lstlisting}
\begin{lstlisting}[caption=Перегрузка оператора вывода]
#include <iostream>
...
ostream& operator<<(ostream& os, const Date& dt)  
{  
    os << dt.mo << '/' << dt.da << '/' << dt.yr;  
    return os;  
}  
\end{lstlisting}
\section{Источники}
\begin{itemize}
\item \url{https://habr.com/post/132014/}
\item \url{https://ru.wikipedia.org/wiki/Полиморфизм\_(информатика)}
\item \url{cppstudio.com/post/310/}
\item \url{https://msdn.microsoft.com/ru-ru/library/5dhe1hce.aspx}
\end{itemize}
\tableofcontents
\end {document}
